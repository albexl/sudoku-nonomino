%%%%%%%%%%%%%%%%%%%%%%%%%%%%%%%%%%%%%%%%%%%%%%%%%%%%%%%%
%							Sudoku Solver Haskell             %
%%%%%%%%%%%%%%%%%%%%%%%%%%%%%%%%%%%%%%%%%%%%%%%%%%%%%%%%

\documentclass[12pt]{article}

\usepackage[latin1]{inputenc}

\usepackage[spanish]{babel}

% Paquetes de la AMS:
\usepackage[total={6in,9.8in},top=0.50in, left=1in, right=1in]{geometry}
\usepackage{amsmath, amsthm, amsfonts}
\usepackage{graphics}
\usepackage{float}
\usepackage{epsfig}
\usepackage{amssymb}
\usepackage{dsfont}
\usepackage{latexsym}
\usepackage{newlfont}
\usepackage{epstopdf}
\usepackage{amsthm}
\usepackage{epsfig}
\usepackage{caption}
\usepackage{multirow}
\usepackage[colorlinks]{hyperref}
\usepackage[x11names,table]{xcolor}
\usepackage{graphics}
\usepackage{wrapfig}
\usepackage[rflt]{floatflt}
\usepackage{multicol}
\usepackage{listings} \lstset {language = haskell, basicstyle=\bfseries\ttfamily, keywordstyle = \color{blue}, commentstyle = \bf\color{brown}}


\renewcommand{\labelenumi}{$\bullet$}

\title{\bf\huge Programaci\'on Declarativa \\Sudoku Nonomino}
\author{\Large Alberto Gonz\'alez Rosales\\
	\large {Grupo C-411}
	}
\date{}

\begin{document}
\maketitle

\section{Problema}
El objetivo de este trabajo es resolver un \emph{sudoku nonomin\'o} utilizando el lenguaje de programaci\'on \emph{Haskell}. Un \emph{sudoku nonomin\'o} es una versi\'on m\'as general del \emph{sudoku cl\'asico}. En este tipo de \emph{sudokus} las reglas son las mismas que en el \emph{sudoku cl\'asico} : se debe rellenar el tablero con n\'umeros entre $1$ y $9$, no debe existir ning\'un n\'umero repetido en ninguna fila ni columna y tampoco en ninguna de las $9$ regiones definidas en el tablero. La diferencia radica en que las regiones de un \emph{sudoku cl\'asico} son cuadrados de $3*3$, mientras que las regiones del \emph{sudoku nonomin\'o} son regiones conexas de $9$ cuadrados de $1*1$.

Nuestro problema espec\'ifico parte del hecho de que se nos da como entrada un conjunto de $9$ \emph{nonomin\'os} y tenemos primeramente que verificar que exista una forma de ubicar las piezas \emph{nonomin\'os} sobre el tablero del \emph{sudoku}(un cuadrado de dimensiones $9*9$). Una vez determinada una posible ubicaci\'on de las piezas sobre el tablero procedemos entonces a resolver el \emph{sudoku nonomin\'o} resultante.

\section{Definiciones e implementaci\'on }
La forma de representar los \emph{nonomin\'os} que se consider\'o m\'as conveniente fue mediante una clase que tuviese dos campos : un identificador num\'erico y una lista de tuplas de la forma $((x, y), val)$, o sea, una posici\'on y un valor asociado a esa posici\'on. El valor $0$ significa que en esa posici\'on a\'un no hay ning\'un n\'umero asignado. El identificador num\'erico sirve para asociar un \'indice a un \emph{nonomin\'o} para que sea m\'as f\'acil identificarlo, lo cual es \'util a la hora de mostrarlos por consola. Los valores $(x, y)$ son las posiciones relativas del \emph{nonomin\'o} asumiendo que el cuadrado m\'as arriba y m\'as a la izquierda es el $(0, 0)$.

La forma de representar un \emph{sudoku} es una lista de tuplas de la forma $((x, y), val)$ donde $(x, y)$ es una posici\'on en la matriz de $9*9$ y $val$ es el valor asociado a esa posici\'on.

La forma de resolver el problema planteado consiste en dada una lista de \emph{nonomin\'os} pasarla como par\'ametro a un m\'etodo que es el encargado de comprobar si existe una distribuci\'on v\'alida de estos y, en caso de encontrarla, pasa esta distribuci\'on a otro m\'etodo que se encarga de resolver el \emph{sudoku}.


\section{Ideas principales para resolver el problema}
El proceso de soluci\'on del \emph{sudoku nonomin\'o} consiste en dos pasos:

\subsection{Ordenar los \emph{nonomin\'os}}
Primeramente hay que comprobar que exista una forma de ubicar los \emph{nonomin\'os} de forma tal que cubran todo el tablero y no exista en esa distribuci\'on ning\'un n\'umero repetido en ninguna fila o columna. Estamos asumiendo que en un mismo \emph{nonomin\'o} no existen n\'umeros repetidos.

Para esto vamos a seguir el criterio de que dado un orden de los \emph{nonomin\'os}, el primero en la lista debe ser el que cubra la posici\'on $(0, 0)$ en la matriz. El siguiente en la lista debe cubrir la posici\'on m\'as arriba y m\'as a la izquierda que a\'un no est\'e cubierta y as\'i sucesivamente. Sabremos si el orden que estamos comprobando es v\'alido si cuando est\'en ubicados todos los \emph{nonomin\'os} no qued\'o ninguna posici\'on vac\'ia.

La idea es comprobar si se puede ubicar alguna de las permutaciones de la lista de \emph{nonomin\'os}, si es posible ubicar los \emph{nonomin\'os} en este orden entonces se pasa a resolver el \emph{sudoku} que qued\'o conformado. En caso de que no exista ninguna distribuci\'on v\'alida para conformar un \emph{sudoku} determinamos que no hay soluci\'on. 

\subsection{Resolver el \emph{sudoku}}
Para resolver el \emph{sudoku} utilizaremos un m\'etodo recursivo que va a comprobar si existe soluci\'on de \emph{nonomin\'o} en \emph{nonomin\'o}. El m\'etodo recibe, a grandes rasgos, el \emph{nonomin\'o} actual, la posici\'on actual dentro de este \emph{nonomin\'o}, el n\'umero que se quiere poner en esta posici\'on y todas las posiciones donde se ha puesto alg\'un n\'umero.

Los casos base son: si revisamos todos los \emph{nonomin\'os} entonces pudimos resolverlo, si el n\'umero que queremos probar en esta posici\'on es mayor o igual a $10$ entonces no hay soluci\'on. De lo contrario comprobamos si el n\'umero que queremos ubicar en la posici\'on actual no genere contradicciones en su fila, columna y \emph{nonomin\'o}. Si no hay contradicci\'on lo ubicamos y resolvemos recursivamente el \emph{sudoku} en la siguiente posici\'on del \emph{nonomin\'o}; si hay contradicci\'on entonces se prueba con el n\'umero actual incrementado en uno.


\section{Funci\'on Main}
La funci\'on que da inicio al programa lee un fichero $"input.in"$ que contiene una lista de \emph{nonomin\'os} en el formato explicado anteriormente. Esta lista de $"Strings"$ la convierte a \emph{nonomin\'os} y la pasa como par\'ametro al m\'etodo principal llamado $"solution"$. Este m\'etodo devuelve una tupla donde el primer elemento es la distribuci\'on de los \emph{nonomin\'os} en el tablero y el segundo es la soluci\'on del \emph{sudoku}. En caso de no existir soluci\'on devuelve dos listas vac\'ias. Posteriormente se muestran en la consola los resultados obtenidos en forma de matrices de $9 * 9$.

Para ejecutar el programa basta con configurar en el archivo $"input.in"$ adjunto, la lista de \emph{nonomin\'os} siguiendo el formato especificado anteriormente y luego ejecutar el comando $"main"$ en la consola una vez cargado el m\'odulo $"nonominos.hs"$.

\subsection{C\'odigo para iniciar el programa}
\begin{lstlisting}
Prelude> :l nonominos.hs
[1 of 1] Compiling Main             ( nonominos.hs, interpreted )
Ok, one module loaded.
*Main> main

\end{lstlisting}

\subsection{C\'odigo de la funci\'on Main}

\begin{lstlisting}
main = do
    inp <- readFile "input.in"              
    let x = map readNonomino (lines inp)   

    let sol = solution x                    

    let setting = fst sol                   
    let filled = snd sol                    

    print "sudoku distribution:"                        
    printList (convertToIds (setting) )   
    
    print "solution:"
    printList (convertToVals (filled) )    


\end{lstlisting}

\subsection{Salida ejemplo}
\begin{lstlisting}

"sudoku distribution:"
[1,1,1,1,1,2,2,2,2]
[3,1,1,1,2,2,2,2,2]
[3,3,1,3,4,4,4,4,4]
[3,3,3,3,6,7,7,4,4]
[5,5,3,6,6,7,7,4,4]
[5,5,6,6,7,7,7,9,9]
[5,5,6,8,8,8,7,9,9]
[5,5,6,8,8,8,7,9,9]
[5,6,6,8,8,8,9,9,9]
"solution:"
[9,5,7,1,8,2,6,3,4]
[6,2,3,4,1,8,7,9,5]
[7,1,6,2,5,3,4,8,9]
[8,3,5,9,7,4,1,2,6]
[2,9,4,5,3,6,8,1,7]
[1,8,2,6,9,7,5,4,3]
[3,7,9,8,4,5,2,6,1]
[4,6,1,7,2,9,3,5,8]
[5,4,8,3,6,1,9,7,2]
\end{lstlisting}

\end{document}
